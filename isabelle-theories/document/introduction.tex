\newpage
\section{Introduction}
 
\noindent
We provide a formal proof of decidability of Galois energy games over vectors of naturals with the component-wise order. 
Afterwards we consider an instantiation of \textit{Bisping's declining energy games}.
Bisping, Nestmann, and Jansen~\cite{bispingNestmann,bispingJansenNestmann} generalised 
Stirling's bisimulation game~\cite{stirling-bisim} to find Hennessy-Milner logic (HML) formulae distinguishing 
processes. Those formalae are elements of some HML-sublanguage from 
van Glabbeek’s linear-time-branching-time spectrum\cite{vanGlabbeek} 
and thus their existence is a statement about behavioural equivalences.
The HML-sublanguages from the linear-time-branching-time spectrum can be characterised by depth properties, 
which can be represented by six-dimensional vectors of extended natural numbers. Understanding these 
vectors as energies Bisping~\cite{bens-algo} developed a multi-weighted energy game deciding all common 
notions of (strong) behavioural equivalences at once, the \textit{spectroscopy game}. 

This game is part of a class of energy games Bisping~\cite{bens-algo} calls \textit{declining energy games}.
Bisping provides an algorithm, which he claims decides this class of energy games if the set of positions is finite.
We substantiate this claim by providing a proof in Isabelle/HOL using a simplyfied and generalised version of that algorithm.
To do so we first formalise energy games with reachability winning conditions in Energy\_Game.thy. 
Building upon this, we then formalise Galois energy games in Galois\_Energy\_Game.thy and prove decidability in Decidability.thy.
Finally, we formalise a superclass of Bisping's declining energy games in Bispings\_Energy\_Game.thy. In particular, we do not assume the games to be declining.
An overview of all our theories is given by the following figure, where the theories above are imported by the ones below.

\begin{figure}[H]
\begin{center}

\definecolor{gray245}{RGB}{245, 245, 245}
\definecolor{color0}{RGB}{0, 0, 0}
\definecolor{color1}{RGB}{51, 51, 51}

\tikzstyle{rect} = [rectangle, minimum width=2.4cm, minimum height=1cm, text centered, font=\normalsize, color=color1, draw=color0, line width=1, fill=gray245]
\tikzstyle{arrowdefi} = [thick, draw=color1, line width=2, ->, >=stealth]

\begin{tikzpicture}[node distance=2cm]
\node (bisping) [state, rect, text width=5cm] {Bispings\_Energy\_Game};
\node (updates) [state, rect, above of=bisping, xshift=+2.8cm, yshift=2cm, text width=4cm] {Update};
\node (order) [state, rect, above of=updates, text width=4cm] {Energy\_Order};
\node (decidable) [state, rect, above of=bisping, xshift=-2.8cm, text width=4cm] {Decidability};
\node (galois) [state, rect, above of=decidable, text width=4cm] {Galois\_Energy\_Game};
\node (games) [state, rect, above of=galois, text width=4cm] {Energy\_Game};
\node (list) [state, rect, above of=order, text width=4cm] {List\_Lemmas};

\draw 
(order) -- (updates)
(order) -- (galois)
(updates) -- (bisping)
(games) -- (galois)
(galois) -- (decidable)
(bisping) -- (decidable)
(list) -- (order)
;
\end{tikzpicture}
\end{center}
\end{figure}

The file List\_Lemmas.thy contains a few simple observations about lists, specifically when using \texttt{those}. This file's contents can be found in the appendix.

Energy games are formalised as two-player zero-sum games with perfect information and reachability winning conditions played on labeled directed graphs in Energy\_Game.thy. 
In particular, strategies and an inductive characterisation of winning budgets is discussed.

In Energy\_Order.thy we introduce the energies, i.e.\ vectors with entries in the extended natural numbers, and the component-wise order. There we establish that this order is a well-founded bounded join-semilattice. 

Galois energy games over such energies with a fixed dimension are formalised in Galois\_energy\_game.thy. 

In Decidability.thy we formalise one iteration of a simplyfied and generalised version of Bisping's algorithm. Using an order on possible Pareto fronts we are able to apply Kleene's fixed point theorem. Assuming the game graph to be finite we then prove correctness of the algorithm. Further, we provide the key argument for termination, thus proving decidability of Galois energy games.

In Update.thy we define a superset of Bisping's updates. These are partial functions of energy vectors updating each component by subtracting or adding one, replacing it with the minimum of some components or not changing it. In particular, we observe that these functions are monotonic and have upward-closed domains.
Further, we introduce a generalisation of Bisping's inversion and relate it to the updates using Galois connections. 

Finally, we formalise energy games where all edges of the game graph are labeled with a representation of the previously discussed updates (and thereby formalise Bisping's declining energy games) in Bispings\_Energy\_Game.thy.
