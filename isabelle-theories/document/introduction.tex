\newpage
\section{Introduction}
 
\noindent
We provide a formal proof of decidability of \textit{Bisping's declining energy games}. 
Bisping, Nestmann, and Jansen \cite{bispingNestmann,bispingJansenNestmann} generalised the 
Stirling's bisimulation game~\cite{stirling-bisim} to find Hennessy-Milner logic (HML) formulae distinguishing 
processes. Those formalae are elements of some HML-sublanguage from 
van Glabbeek’s linear-time-branching-time spectrum\cite{vanGlabbeek} 
and thus their existence is a statement about behavioural equivalences.
The HML-sublanguages from the linear-time-branching-time spectrum can be characterised by depth properties, 
which can be represented by six-dimensional vectors of extended natural numbers. Understanding these 
vectors as energies Bisping\cite{bens-algo} developed a multi-weighted energy game deciding all common 
notions of (strong) behavioural equivalences at once, the \textit{spectroscopy game}. 

This game is part of a class of energy games Bisping~\cite{bens-algo} calls \textit{declining energy games}.
Bisping provides an algorithm, which he claims decides this class of energy games if the set of positions is finite.
We substantiate this claim by providing a proof in Isabelle/HOL.
To do so we first formalise energy games with reachability winning conditions in Energy\_Game.thy. 
Building upon this, we then formalise Bisping's declining energy games in Bispings\_Energy\_Game.thy and prove decidability in Decidability.thy.
An overview of all our theories is given by the following figure, where the theories above are imported by the ones below.

\begin{figure}[H]
\begin{center}

\definecolor{gray245}{RGB}{245, 245, 245}
\definecolor{color0}{RGB}{0, 0, 0}
\definecolor{color1}{RGB}{51, 51, 51}

\tikzstyle{rect} = [rectangle, minimum width=2.4cm, minimum height=1cm, text centered, font=\normalsize, color=color1, draw=color0, line width=1, fill=gray245]
\tikzstyle{arrowdefi} = [thick, draw=color1, line width=2, ->, >=stealth]

\begin{tikzpicture}[node distance=2cm]
\node (bisping) [state, rect, text width=4.5cm] {Bispings\_Energy\_Game};
\node (games) [state, rect, above of=bisping, xshift=-2.4cm, text width=3.5cm] {Energy\_Game};
\node (updates) [state, rect, above of=bisping, xshift=+2.4cm, text width=3.5cm] {Update};
\node (order) [state, rect, above of=updates, text width=3.5cm] {Energy\_Order};
\node (decidable) [state, rect, below of=bisping, text width=4.5cm] {Decidability};
\node (list) [state, rect, above of=order, text width=3.5cm] {List\_Lemmas};

\draw 
(order) -- (updates)
(updates) -- (bisping)
(games) -- (bisping)
(bisping) -- (decidable)
(list) -- (order)
;
\end{tikzpicture}
\end{center}
\caption{Extract from session graph}\label{fig:sessiongraph}
\end{figure}

Energy games are formalised as two-player zero-sum games with perfect information and reachability winning conditions played on labeled directed graphs in Energy\_Game.thy. 
In particular, strategies and an inductive characterisation of winning budgets is discussed.

The file List\_Lemmas.thy contains a few simple observations about list, specifically when using \texttt{those}. This file's contents can be found in the appendix.

In Energy\_Order.thy we introduce the energies, i.e.\ vectors with entries in the extended natural number, and the component-wise order. There we establish that this order is a well-founded join-semilattice. 

In Update.thy we then definie Bisping's updates. These are partial functions of energy vectors updating each component by subtracting one, replacing it with the minimum of itself and some components or not changing it. In particular, we observe that these functions are declining and have upward-closed domains.
Further, we introduce Bisping's inversion and relate it to Bisping's updates using Galois connections. 

Bisping's declining energy games with a fixed dimension are formalised in Bispings\_Energy\_Game.thy. In these games edges of the game graph are labeled with a representation of the previously discussed updates. 

In Decidability.thy we formalise one iteration of a simplification of Bisping's algorithm. Using an order on possible Pareto fronts we are able to apply Kleene's fixed point theorem. Assuming the game graph to be finite we thereby prove correctness of the algorithm. Further, we provide the key argument for termination, thus proving decidability. 
